% Options for packages loaded elsewhere
\PassOptionsToPackage{unicode}{hyperref}
\PassOptionsToPackage{hyphens}{url}
\PassOptionsToPackage{dvipsnames,svgnames,x11names}{xcolor}
%
\documentclass[
  letterpaper,
  DIV=11,
  numbers=noendperiod]{scrreprt}

\usepackage{amsmath,amssymb}
\usepackage{iftex}
\ifPDFTeX
  \usepackage[T1]{fontenc}
  \usepackage[utf8]{inputenc}
  \usepackage{textcomp} % provide euro and other symbols
\else % if luatex or xetex
  \usepackage{unicode-math}
  \defaultfontfeatures{Scale=MatchLowercase}
  \defaultfontfeatures[\rmfamily]{Ligatures=TeX,Scale=1}
\fi
\usepackage{lmodern}
\ifPDFTeX\else  
    % xetex/luatex font selection
\fi
% Use upquote if available, for straight quotes in verbatim environments
\IfFileExists{upquote.sty}{\usepackage{upquote}}{}
\IfFileExists{microtype.sty}{% use microtype if available
  \usepackage[]{microtype}
  \UseMicrotypeSet[protrusion]{basicmath} % disable protrusion for tt fonts
}{}
\makeatletter
\@ifundefined{KOMAClassName}{% if non-KOMA class
  \IfFileExists{parskip.sty}{%
    \usepackage{parskip}
  }{% else
    \setlength{\parindent}{0pt}
    \setlength{\parskip}{6pt plus 2pt minus 1pt}}
}{% if KOMA class
  \KOMAoptions{parskip=half}}
\makeatother
\usepackage{xcolor}
\setlength{\emergencystretch}{3em} % prevent overfull lines
\setcounter{secnumdepth}{5}
% Make \paragraph and \subparagraph free-standing
\makeatletter
\ifx\paragraph\undefined\else
  \let\oldparagraph\paragraph
  \renewcommand{\paragraph}{
    \@ifstar
      \xxxParagraphStar
      \xxxParagraphNoStar
  }
  \newcommand{\xxxParagraphStar}[1]{\oldparagraph*{#1}\mbox{}}
  \newcommand{\xxxParagraphNoStar}[1]{\oldparagraph{#1}\mbox{}}
\fi
\ifx\subparagraph\undefined\else
  \let\oldsubparagraph\subparagraph
  \renewcommand{\subparagraph}{
    \@ifstar
      \xxxSubParagraphStar
      \xxxSubParagraphNoStar
  }
  \newcommand{\xxxSubParagraphStar}[1]{\oldsubparagraph*{#1}\mbox{}}
  \newcommand{\xxxSubParagraphNoStar}[1]{\oldsubparagraph{#1}\mbox{}}
\fi
\makeatother


\providecommand{\tightlist}{%
  \setlength{\itemsep}{0pt}\setlength{\parskip}{0pt}}\usepackage{longtable,booktabs,array}
\usepackage{calc} % for calculating minipage widths
% Correct order of tables after \paragraph or \subparagraph
\usepackage{etoolbox}
\makeatletter
\patchcmd\longtable{\par}{\if@noskipsec\mbox{}\fi\par}{}{}
\makeatother
% Allow footnotes in longtable head/foot
\IfFileExists{footnotehyper.sty}{\usepackage{footnotehyper}}{\usepackage{footnote}}
\makesavenoteenv{longtable}
\usepackage{graphicx}
\makeatletter
\def\maxwidth{\ifdim\Gin@nat@width>\linewidth\linewidth\else\Gin@nat@width\fi}
\def\maxheight{\ifdim\Gin@nat@height>\textheight\textheight\else\Gin@nat@height\fi}
\makeatother
% Scale images if necessary, so that they will not overflow the page
% margins by default, and it is still possible to overwrite the defaults
% using explicit options in \includegraphics[width, height, ...]{}
\setkeys{Gin}{width=\maxwidth,height=\maxheight,keepaspectratio}
% Set default figure placement to htbp
\makeatletter
\def\fps@figure{htbp}
\makeatother
% definitions for citeproc citations
\NewDocumentCommand\citeproctext{}{}
\NewDocumentCommand\citeproc{mm}{%
  \begingroup\def\citeproctext{#2}\cite{#1}\endgroup}
\makeatletter
 % allow citations to break across lines
 \let\@cite@ofmt\@firstofone
 % avoid brackets around text for \cite:
 \def\@biblabel#1{}
 \def\@cite#1#2{{#1\if@tempswa , #2\fi}}
\makeatother
\newlength{\cslhangindent}
\setlength{\cslhangindent}{1.5em}
\newlength{\csllabelwidth}
\setlength{\csllabelwidth}{3em}
\newenvironment{CSLReferences}[2] % #1 hanging-indent, #2 entry-spacing
 {\begin{list}{}{%
  \setlength{\itemindent}{0pt}
  \setlength{\leftmargin}{0pt}
  \setlength{\parsep}{0pt}
  % turn on hanging indent if param 1 is 1
  \ifodd #1
   \setlength{\leftmargin}{\cslhangindent}
   \setlength{\itemindent}{-1\cslhangindent}
  \fi
  % set entry spacing
  \setlength{\itemsep}{#2\baselineskip}}}
 {\end{list}}
\usepackage{calc}
\newcommand{\CSLBlock}[1]{\hfill\break\parbox[t]{\linewidth}{\strut\ignorespaces#1\strut}}
\newcommand{\CSLLeftMargin}[1]{\parbox[t]{\csllabelwidth}{\strut#1\strut}}
\newcommand{\CSLRightInline}[1]{\parbox[t]{\linewidth - \csllabelwidth}{\strut#1\strut}}
\newcommand{\CSLIndent}[1]{\hspace{\cslhangindent}#1}

\KOMAoption{captions}{tableheading}
\makeatletter
\@ifpackageloaded{bookmark}{}{\usepackage{bookmark}}
\makeatother
\makeatletter
\@ifpackageloaded{caption}{}{\usepackage{caption}}
\AtBeginDocument{%
\ifdefined\contentsname
  \renewcommand*\contentsname{Table of contents}
\else
  \newcommand\contentsname{Table of contents}
\fi
\ifdefined\listfigurename
  \renewcommand*\listfigurename{List of Figures}
\else
  \newcommand\listfigurename{List of Figures}
\fi
\ifdefined\listtablename
  \renewcommand*\listtablename{List of Tables}
\else
  \newcommand\listtablename{List of Tables}
\fi
\ifdefined\figurename
  \renewcommand*\figurename{Figure}
\else
  \newcommand\figurename{Figure}
\fi
\ifdefined\tablename
  \renewcommand*\tablename{Table}
\else
  \newcommand\tablename{Table}
\fi
}
\@ifpackageloaded{float}{}{\usepackage{float}}
\floatstyle{ruled}
\@ifundefined{c@chapter}{\newfloat{codelisting}{h}{lop}}{\newfloat{codelisting}{h}{lop}[chapter]}
\floatname{codelisting}{Listing}
\newcommand*\listoflistings{\listof{codelisting}{List of Listings}}
\makeatother
\makeatletter
\makeatother
\makeatletter
\@ifpackageloaded{caption}{}{\usepackage{caption}}
\@ifpackageloaded{subcaption}{}{\usepackage{subcaption}}
\makeatother

\ifLuaTeX
  \usepackage{selnolig}  % disable illegal ligatures
\fi
\usepackage{bookmark}

\IfFileExists{xurl.sty}{\usepackage{xurl}}{} % add URL line breaks if available
\urlstyle{same} % disable monospaced font for URLs
\hypersetup{
  pdftitle={Análise do Cumprimento das Metas de Redução de Mortes no Trânsito dos Municípios Brasileiros até 2023.},
  pdfauthor={Ana Beatriz da S. Marques; Dr.~Jorge Tiago Bastos},
  colorlinks=true,
  linkcolor={blue},
  filecolor={Maroon},
  citecolor={Blue},
  urlcolor={Blue},
  pdfcreator={LaTeX via pandoc}}


\title{Análise do Cumprimento das Metas de Redução de Mortes no Trânsito
dos Municípios Brasileiros até 2023.}
\author{Ana Beatriz da S. Marques \and Dr.~Jorge Tiago Bastos}
\date{2025-07-30}

\begin{document}
\maketitle

\renewcommand*\contentsname{Table of contents}
{
\hypersetup{linkcolor=}
\setcounter{tocdepth}{2}
\tableofcontents
}

\bookmarksetup{startatroot}

\chapter*{Resumo}\label{resumo}
\addcontentsline{toc}{chapter}{Resumo}

\markboth{Resumo}{Resumo}

\begin{center}\rule{0.5\linewidth}{0.5pt}\end{center}

O Plano Nacional de Redução de Mortes e Lesões no Trânsito (PNATRANS),
criado pela Lei Federal nº 13.614/2018, tem como objetivo estabelecer
metas de redução da mortalidade no trânsito para os estados e para o
país entre 2019 e 2028. Em 2021, o plano foi revisado, prorrogando o
prazo das metas até 2030 e reformulando seus pilares de atuação. As
metas de redução foram definidas para os municípios brasileiros através
das técnicas de \emph{clusterização} e \emph{benchmarking}. O objetivo,
agora, é observar o desempenho de cada município até 2023, atráves da
visualização de dados. Para isso, foi criado um Dashboard que contém
todas as informações necessárias para essa análise. O DashBoard para
visualização do desempenho dos municípios está disponível neste link:
\href{https://beatrizmarques.shinyapps.io/cumprimento_metas/}{Dashboard}.

\bookmarksetup{startatroot}

\chapter*{Highlights}\label{highlights}
\addcontentsline{toc}{chapter}{Highlights}

\markboth{Highlights}{Highlights}

\begin{center}\rule{0.5\linewidth}{0.5pt}\end{center}

\begin{itemize}
\tightlist
\item
  35\% dos municípios brasileiros já atingiram a meta estabelecida;
\item
  O estado do Acre obteve o melhor desempenho, com uma redução de 15\%
  das mortes no trânsito e 50\% dos seus municípios com a meta já
  atingida;
\item
  Das 27 unidades da federação, 22 obteviveram um aumento no número de
  mortes no trânsito em 2023;
\item
  Alguns estados apresentaram aumento nas mortes de até 26,45\%,
  indicando um agravamento na situação de segurança viária;
\item
  São Paulo foi a capital com melhor desempenho, registrando uma redução
  de 52\% de mortes no trânsito;
\item
  Em 2023, o Brasil registrou 34599 mortes no trânsito, apontando um
  aumento de 6,66\% em relação à 2020;
\end{itemize}

\bookmarksetup{startatroot}

\chapter*{Sobre o Observatório}\label{sobre-o-observatuxf3rio}
\addcontentsline{toc}{chapter}{Sobre o Observatório}

\markboth{Sobre o Observatório}{Sobre o Observatório}

\begin{center}\rule{0.5\linewidth}{0.5pt}\end{center}

O Observatório Nacional de Segurança Viária é uma instituição social sem
fins lucrativos, dedicada a desenvolver ações que contribuam
efetivamente para a redução dos elevados índices de ocorrências no
trânsito brasileiro. Com esse objetivo, um grupo de profissionais
multidisciplinares decidiu reunir todo o seu conhecimento, experiência e
motivação em um único projeto grandioso e desafiador: mobilizar a
sociedade em prol de um trânsito mais seguro.

\bookmarksetup{startatroot}

\chapter*{Como citar}\label{como-citar}
\addcontentsline{toc}{chapter}{Como citar}

\markboth{Como citar}{Como citar}

\begin{center}\rule{0.5\linewidth}{0.5pt}\end{center}

ONSV; UFPR (2025). Análise do Cumprimento das Metas de Redução de Mortes
no Trânsito dos Municípios Brasileiros até 2023. Observatório Nacional
de Segurança Viária e Universidade Federal do Paraná. Disponível em:
https://onsv.github.io/relatorio-cumprimento-metas/

\begin{figure}

\begin{minipage}{0.36\linewidth}
\begin{center}
\includegraphics{img/onsv-logo.jpg}
\end{center}
\end{minipage}%
%
\begin{minipage}{0.27\linewidth}
\begin{center}
\includegraphics{img/marca_UFPR.png}
\end{center}
\end{minipage}%

\end{figure}%

```

\bookmarksetup{startatroot}

\chapter{Introdução}\label{introduuxe7uxe3o}

O Plano Nacional de Redução de Mortes e Lesões no Trânsito (PNATRANS),
instituído pela Lei nº 13.614/2018, tem como objetivo central reduzir em
pelo menos 50\% o número de mortes no trânsito brasileiro ao longo de
uma década. Após dificuldades iniciais de implementação, o plano foi
revisado em 2021, com participação de mais de 100 especialistas e
representantes de diversas instituições públicas e privadas, resultando
em uma versão mais moderna e aplicável, estruturada em seis pilares
temáticos. A nova versão do PNATRANS incorporou as abordagens de
``Sistemas Seguros'' e ``Visão Zero'', adaptadas à realidade brasileira,
e passou a enfatizar o papel estratégico dos municípios na promoção de
ações locais de segurança viária.

Dentre os avanços trazidos pela revisão, destaca-se a elaboração de
indicadores de desempenho e a proposição de metas específicas de redução
de mortes no trânsito para cada município, considerando as
particularidades regionais. Esses elementos viabilizam a atuação baseada
em evidências e permitem um acompanhamento mais preciso do progresso em
diferentes territórios.

Nesse contexto, este novo estudo tem como objetivo realizar o
monitoramento das metas de redução de mortes no trânsito com base nos
dados mais recentes, referentes ao ano de 2023. A proposta é avaliar o
desempenho dos municípios frente às metas estabelecidas, identificar
quais localidades avançaram significativamente na redução da mortalidade
viária, quais ainda enfrentam desafios relevantes e, assim, contribuir
para o fortalecimento da gestão local da segurança no trânsito. Esse
acompanhamento contínuo é essencial para garantir que os esforços
empreendidos estejam alinhados com os objetivos do PNATRANS e para
promover os ajustes necessários na direção de um trânsito mais seguro
para todos.

\bookmarksetup{startatroot}

\chapter{Metodologia}\label{metodologia}

A unidade de análise adotada foram os municípios brasileiros que
possuíam registros de mortes no trânsito entre 2018 e 2020. Do total de
municípios existentes no Brasil em 2021 (5.570), 5.044 apresentaram
registros de óbitos por sinistros viários nesse período. No entanto,
para assegurar a robustez da análise, foram excluídos os municípios
considerados outliers, restando um total de 4.473 municípios com dados
considerados confiáveis. As metas de redução foram estabelecidas
exclusivamente para esses municípios não outliers e por isso, neste
estudo foi realizada a comparação apenas destes municípios.

A principal base utilizada para atualização dos dados foi a
\texttt{roadtrafficdeaths}, em que \textbf{cada linha representa uma
morte por sinistro de trânsito}, contendo informações detalhadas como
data, localização e características da ocorrência. A partir dessa base,
foi realizado um \textbf{processo de agregação por município},
contabilizando o total de óbitos registrados no ano de 2023.

Depois, as bases de frota e número habitantes da população de cada
município foram agregadas à essa base, com o objetivo de facilitar a
análise da tabela interativa do dashboard.

Em seguida, os dados agregados foram integrados à base de metas do
estudo original (realizado entre 2020 e 2021). A comparação da redução
de mortes no trânsito em 2023 em relação às metas de redução foram
calculadas da seguinte forma para cada município:

Os cálculos gerais do dashboard (para cada estado e para o Brasil),
foram calculados utilizando dados de todos os municípios, incluindo os
que não tinham metas estabelecidas estabelecidas pelo PNATRANS. \[
\text{Redução} = \frac{\text{Número de mortes (2023)} - \text{Média de mortes (2018–2020)}}{\text{Média de mortes (2018–2020)}}
\]

\[
\text{Percentual da meta atingida} = \left( \frac{\text{Redução}}{\text{Meta estabelecida}} \right) \times 100
\]

A análise foi realizada utilizando a linguagem R, e os resultados foram
sistematizados em um dashboard desenvolvido com o pacote \texttt{Shiny}.
A ferramenta permite a visualização interativa dos resultados por
município e por estado, incluindo gráficos comparativos e tabelas, com o
intuito de facilitar a visualização do cumprimento das metas.

\bookmarksetup{startatroot}

\chapter{Resultados}\label{resultados}

\section{Resumo}\label{resumo-1}

Com o objetivo de tornar os resultados deste estudo acessíveis de forma
interativa e transparente, foi desenvolvido um \textbf{dashboard em R
utilizando o pacote Shiny}, disponível no seguinte endereço:

\url{https://beatrizmarques.shinyapps.io/cumprimento_metas/}

O painel permite a visualização detalhada do \textbf{cumprimento das
metas de redução de mortes no trânsito} estabelecidas pelo PNATRANS para
os municípios brasileiros classificados como não outliers (n = 4.473).

A interface é organizada por \textbf{unidades da federação}, e cada aba
apresenta:

\begin{itemize}
\item
  \textbf{Gráfico de dispersão}: mostra a variação proporcional do
  número de mortes em 2023 em relação à média de 2018--2020, destacando
  os municípios que atingiram ou não suas metas;
\item
  \textbf{Gráfico de barras}: exibe o percentual de municípios que
  cumpriram as metas em cada estado e capital;
\item
  ️ \textbf{Tabela interativa}: apresenta, para cada município,
  informações como população, frota, número de mortes em 2023, meta
  individualizada e se foi atingida;
\end{itemize}

O dashboard foi desenvolvido para ser uma \textbf{ferramenta de
visualização do cumprimento das metas}, possibilitando a identificação
de áreas críticas e de boas práticas locais.

\section{Abas}\label{abas}

O dashboard foi estruturado com um menu lateral que organiza o conteúdo
por \textbf{unidades da federação}, facilitando a navegação e a análise
regionalizada do cumprimento das metas.

Cada aba representa um estado brasileiro (ou o consolidado nacional
``Brasil'') e, ao ser selecionada, exibe um conjunto de visualizações e
informações específicas para os municípios daquela unidade federativa.
Essa estrutura permite ao usuário realizar consultas direcionadas,
comparações regionais e identificar padrões de desempenho na redução de
mortes no trânsito.

\section{Conteúdo}\label{conteuxfado}

Para cada aba (UF), o conteúdo apresentado inclui:

\begin{itemize}
\item
  Média de mortes no trânsito (2018-2020)
\item
  Número de mortes no trânsito (2023);
\item
  Percentual de municípios que já atingiram a meta;
\item
  Total de municípios contabilizados no cálculo (não outliers);
\item
  Qual era a meta estabelecida para o total;
\item
  Percentual de redução ou aumento no número de mortes;
\item
  Gráfico de barras por estado;
\item
  Gráfico de barras por capital;
\item
  Gráfico de dispersão: meta estabelecida x meta atingida;
\item
  Tabela interativa.
\end{itemize}

\subsection{Gráfico de Barras}\label{gruxe1fico-de-barras}

Os gráficos de barras mostram o desempenho de cada estado e capital na
redução das mortes no trânsito. Por serem interativos, é possível passar
o cursor sobre as barras para visualizar o desempenho detalhado de cada
unidade da federação. As barras à esquerda de 0\% indicam redução no
número de mortes, enquanto as à direita indicam aumento em 2023. Em cada
aba, as barras do estado e da respectiva capital ficam em destaque para
facilitar a visualização.

\subsection{Gráfico de Dispersão}\label{gruxe1fico-de-dispersuxe3o}

O gráfico de dispersão representa o percentual de alcance da meta em
relação ao valor estabelecido. O eixo X exibe a meta de redução, sempre
expressa em valores negativos. O eixo Y indica o percentual já atingido
por cada município: valores negativos correspondem a um desempenho
abaixo do esperado (contrário à meta), enquanto valores positivos
indicam progresso. Cada ponto representa um município, com a cor
vermelha indicando aumento no número de mortes e a verde, redução. Ao
passar o cursor sobre os pontos, são exibidas informações detalhadas:
nome do município, estado, meta de redução, percentual de redução, média
de mortes (2018--2020) e número de mortes em 2023.

\subsection{Tabela}\label{tabela}

A Tabela apresenta a lista de todos os municípios que tinham uma meta
estabelecida. Nela, é possível verificar todas as informações de cada
município, incluindo o total da frota e o número de habitantes em 2023.
Os municípios estão ordenados por prioridade (1-27) e é possível
pesquisar um município específico através da barra de pesquisa no canto
direito superior da tabela. Para facilitar a análise, há também um
filtro que permite visualizar apenas os municípios que já atingiram a
meta ou os que não atingiram.

\bookmarksetup{startatroot}

\chapter{Conclusão}\label{conclusuxe3o}

Os resultados apresentados evidenciam a complexidade e os desafios
enfrentados pelos municípios brasileiros no cumprimento das metas de
redução de mortes no trânsito até 2023. Apesar de iniciativas e esforços
voltados à segurança viária, os dados revelam que apenas uma parcela dos
municípios conseguiu atingir ou superar as metas estabelecidas, enquanto
muitos ainda apresentam níveis de mortalidade superiores à média do
período-base (2018--2020).

A análise regional mostra disparidades significativas entre estados e
municípios, sugerindo que fatores locais, como infraestrutura,
fiscalização, políticas públicas e educação para o trânsito, influenciam
diretamente os resultados obtidos. Essas variações reforçam a
importância de ações direcionadas, adaptadas à realidade de cada
localidade.

O dashboard se consolida como uma ferramenta valiosa de monitoramento e
transparência, permitindo que gestores públicos, pesquisadores e a
sociedade acompanhem de forma clara e objetiva o progresso das metas
pactuadas. Mais do que um diagnóstico, ele oferece subsídios para a
tomada de decisão e o aprimoramento das estratégias de prevenção e
redução de mortes no trânsito.

\section{Brasil}\label{brasil}

\begin{itemize}
\item
  Dos 5.570 municípios do Brasil, 5.044 municípios não outliers tiveram
  uma meta estabelecida pelo PNATRANS em 2021.
\item
  O Brasil registrou, no total, 34.881 mortes no trânsito em 2023,
  apontando um aumento de 7,53\% em relação aos anos 2018-2020, que
  tiveram uma média de 32.437,33 mortes no trânsito.
\item
  A meta de redução para o Brasil era de -37,96\%.
\item
  1.771 municípios já atingiram a meta, um percentual de 35,11\% do
  total de municípios com meta estabelecida.
\end{itemize}

\section{Acre}\label{acre}

\begin{itemize}
\item
  Dos 22 municípios do Acre, 18 municípios não outliers tiveram uma meta
  estabelecida pelo PNATRANS em 2021.
\item
  A capital do Acre, Rio Branco, apresentou uma redução de -8,56\% no
  número de mortes em 2023.
\item
  O estado do Acre registrou, no total, 93 mortes no trânsito em 2023,
  apontando uma redução de -13,62\% em relação aos anos 2018-2020, que
  tiveram uma média de 107,67 mortes no trânsito.
\item
  A meta de redução para o Acre era de -55,43\%.
\item
  9 municípios já atingiram a meta, um percentual de 50\% do total de
  muncípios.
\end{itemize}

\section{Alagoas}\label{alagoas}

\begin{itemize}
\item
  A capital de Alagoas, Maceió, apresentou uma redução de -6,49\% no
  número de mortes em 2023.
\item
  Dos 102 municípios de Alagoas, 97 municípios não outliers tiveram uma
  meta estabelecida pelo PNATRANS em 2021.
\item
  O estado de Alagoas registrou, no total, 626 mortes no trânsito em
  2023, apontando uma redução de -0,42\% em relação aos anos 2018-2020,
  que tiveram uma média de 628,67 mortes no trânsito.
\item
  A meta de redução para Alagoas era de -42,33\%.
\item
  36 municípios já atingiram a meta, um percentual de 37,11\% do total
  de municípios com meta estabelecida.
\end{itemize}

\section{Amapá}\label{amapuxe1}

\begin{itemize}
\item
  A capital do Amapá, Macapá, apresentou um aumento de 8,77\% no número
  de mortes em 2023.
\item
  Dos 16 municípios do Amapá, 12 municípios não outliers tiveram uma
  meta estabelecida pelo PNATRANS em 2021.
\item
  O estado do Amapá registrou, no total, 88 mortes no trânsito em 2023,
  apontando um aumento de 19,46\% em relação aos anos 2018-2020, que
  tiveram uma média de 73,67 mortes no trânsito.
\item
  A meta de redução para o Amapá era de -22,52\%.
\item
  5 municípios já atingiram a meta, um percentual de 41,67\% do total de
  municípios com meta estabelecida.
\end{itemize}

\section{Amazonas}\label{amazonas}

\begin{itemize}
\item
  A capital do Amazonas, Manaus, apresentou um aumento de 12,68\% no
  número de mortes em 2023.
\item
  Dos 62 municípios do Amapá, apenas 44 municípios não outliers tiveram
  uma meta estabelecida pelo PNATRANS em 2021.
\item
  O estado do Amazonas registrou, no total, 439 mortes no trânsito em
  2023, apontando um aumento de 9,11\% em relação aos anos 2018-2020,
  que tiveram uma média de 402,33 mortes no trânsito.
\item
  A meta de redução para o Amazonas era de -73,36\%.
\item
  15 municípios já atingiram a meta, um percentual de 34,09\% do total
  de municípios com meta estabelecida.
\end{itemize}

\section{Bahia}\label{bahia}

\begin{itemize}
\item
  A capital da Bahia, Salvador, apresentou uma redução de de -5,95\% no
  número de mortes em 2023.
\item
  Dos 417 municípios do Amapá, foram considerados 403 municípios não
  outliers e que tiveram uma meta estabelecida pelo PNATRANS em 2021.
\item
  O estado de registrou, no total, 2.838 mortes no trânsito em 2023,
  apontando um aumento de 27,07\% em relação aos anos 2018-2020, que
  tiveram uma média de 2.233,33 mortes no trânsito.
\item
  A meta de redução para a Bahia era de -42,29\%.
\item
  108 municípios já atingiram a meta, um percentual de 26,80\% do total
  de municípios com meta estabelecida.
\end{itemize}

\section{Ceará}\label{cearuxe1}

\begin{itemize}
\item
  A capital do Ceará, Fortaleza, apresentou um aumento de 4,74\% no
  número de mortes em 2023.
\item
  Dos 184 municípios do Ceará, foram considerados 181 municípios não
  outliers e que tiveram uma meta estabelecida pelo PNATRANS em 2021.
\item
  O estado de registrou, no total, 1.374 mortes no trânsito em 2023,
  apontando uma redução de 2,78\% em relação aos anos 2018-2020, que
  tiveram uma média de 1413,33 mortes no trânsito.
\item
  A meta de redução para o Amapá era de -43,70\%.
\item
  59 municípios já atingiram a meta, um percentual de 32,60\% do total
  de municípios com meta estabelecida.
\end{itemize}

\section{Distrito Federal}\label{distrito-federal}

\begin{itemize}
\item
  Brasília, único município do Distrito Federal, registrou, no total,
  315 mortes no trânsito em 2023, apontando uma redução de 8,25\% em
  relação aos anos 2018-2020, que tiveram uma média de 343,33 mortes no
  trânsito.
\item
  A meta de redução para o Distrito Federal era de -6,47\%.
\item
  Brasília já atingiu a meta, ou seja, 100\% do total de municípios do
  Distrito Federal.
\end{itemize}

\section{Espirito Santo}\label{espirito-santo}

\begin{itemize}
\item
  A capital do Espírito Santo, Vitória, apresentou uma redução de
  -5,52\% no número de mortes em 2023.
\item
  Dos 78 municípios do Espírito Santo, foram considerados 76 municípios
  não outliers e que tiveram uma meta estabelecida pelo PNATRANS em
  2021.
\item
  O estado do Espírito Santo registrou, no total, 839 mortes no trânsito
  em 2023, apontando um aumento de 9,91\% em relação aos anos 2018-2020,
  que tiveram uma média de 763,33 mortes no trânsito.
\item
  A meta de redução para o Espírito Santo era de -43,97\%.
\item
  17 municípios já atingiram a meta, um percentual de 22,37\% do total
  de municípios com meta estabelecida.
\end{itemize}

\section{Goiás}\label{goiuxe1s}

\begin{itemize}
\item
  A capital de Goiás, Goiânia, apresentou um aumento de 17,40\% no
  número de mortes em 2023.
\item
  Dos 246 municípios de Goiás, foram considerados 226 municípios não
  outliers e que tiveram uma meta estabelecida pelo PNATRANS em 2021.
\item
  O estado de Goiás registrou, no total, 1.709 mortes no trânsito em
  2023, apontando um aumento de 11,92\% em relação aos anos 2018-2020,
  que tiveram uma média de 1.527 mortes no trânsito.
\item
  A meta de redução para Goiás era de -44,27\%.
\item
  70 municípios já atingiram a meta, um percentual de 30,97\% do total
  de municípios com meta estabelecida.
\end{itemize}

\section{Maranhão}\label{maranhuxe3o}

\begin{itemize}
\item
  A capital do Maranhão, São Luís, apresentou um aumento de 16,13\% no
  número de mortes em 2023.
\item
  Dos 217 municípios do Maranhão, foram considerados 212 municípios não
  outliers e que tiveram uma meta estabelecida pelo PNATRANS em 2021.
\item
  O estado do Maranhão registrou, no total, 1.444 mortes no trânsito em
  2023, apontando um aumento de 7,73\% em relação aos anos 2018-2020,
  que tiveram uma média de 1.340,33 mortes no trânsito.
\item
  A meta de redução para o Maranhão era de -43,28\%.
\item
  63 municípios já atingiram a meta, um percentual de 29,72\% do total
  de municípios com meta estabelecida.
\end{itemize}

\section{Mato Grosso}\label{mato-grosso}

\begin{itemize}
\item
  A capital do Mato Grosso, Cuiabá, apresentou uma redução de -5,01\% no
  número de mortes em 2023.
\item
  Dos 142 municípios do Mato Grosso, foram considerados 131 municípios
  não outliers e que tiveram uma meta estabelecida pelo PNATRANS em
  2021.
\item
  O estado do Mato Grosso registrou, no total, 1.278 mortes no trânsito
  em 2023, apontando um aumento de 18,85\% em relação aos anos
  2018-2020, que tiveram uma média de 1.075,33 mortes no trânsito.
\item
  A meta de redução para o Mato Grosso era de -59,03\%.
\item
  26 municípios já atingiram a meta, um percentual de 19,85\% do total
  de municípios com meta estabelecida.
\end{itemize}

\section{Mato Grosso do Sul}\label{mato-grosso-do-sul}

\begin{itemize}
\item
  A capital do Mato Grosso do Sul, Campo Grande, apresentou um aumento
  de 5,42\% no número de mortes em 2023.
\item
  Dos 79 municípios do Mato Grosso do Sul, foram considerados 77
  municípios não outliers e que tiveram uma meta estabelecida pelo
  PNATRANS em 2021.
\item
  O estado do Mato Grosso do Sul registrou, no total, 718 mortes no
  trânsito em 2023, apontando um aumento de 18,35\% em relação aos anos
  2018-2020, que tiveram uma média de 606,67 mortes no trânsito.
\item
  A meta de redução para o Mato Grosso do Sul era de -33,42\%.
\item
  16 municípios já atingiram a meta, um percentual de 20,78\% do total
  de municípios com meta estabelecida.
\end{itemize}

\section{Minas Gerais}\label{minas-gerais}

\begin{itemize}
\item
  A capital de Minas Gerais, Belo Horizonte, apresentou um aumento de
  5,45\% no número de mortes em 2023.
\item
  Dos 853 municípios de Minas Gerais, foram considerados 722 municípios
  não outliers e que tiveram uma meta estabelecida pelo PNATRANS em
  2021.
\item
  O estado de Minas Gerais registrou, no total, 3.350 mortes no trânsito
  em 2023, apontando um aumento de 5,6\% em relação aos anos 2018-2020,
  que tiveram uma média de 3.172,33 mortes no trânsito.
\item
  A meta de redução para Minas Gerais era de -36,63\%.
\item
  316 municípios já atingiram a meta, um percentual de 43,77\% do total
  de municípios com meta estabelecida.
\end{itemize}

\section{Pará}\label{paruxe1}

\begin{itemize}
\item
  A capital do Pará, Belém, apresentou uma redução de -9,89\% no número
  de mortes em 2023.
\item
  Dos 144 municípios do Pará, foram considerados 134 municípios não
  outliers e que tiveram uma meta estabelecida pelo PNATRANS em 2021.
\item
  O estado do Pará registrou, no total, 1.601 mortes no trânsito em
  2023, apontando um aumento de 11,70\% em relação aos anos 2018-2020,
  que tiveram uma média de 1.433,33 mortes no trânsito.
\item
  A meta de redução para o Pará era de -56,40\%.
\item
  28 municípios já atingiram a meta, um percentual de 20,90\% do total
  de municípios com meta estabelecida.
\end{itemize}

\section{Paraíba}\label{parauxedba}

\begin{itemize}
\item
  A capital da Paraíba, João Pessoa, apresentou uma redução de -6,07\%
  no número de mortes em 2023.
\item
  Dos 223 municípios da Paraíba, foram considerados 203 municípios não
  outliers e que tiveram uma meta estabelecida pelo PNATRANS em 2021.
\item
  O estado da Paraíba registrou, no total, 840 mortes no trânsito em
  2023, apontando um aumento de 4\% em relação aos anos 2018-2020, que
  tiveram uma média de 807,67 mortes no trânsito.
\item
  A meta de redução para a Paraíba era de -43,21\%.
\item
  75 municípios já atingiram a meta, um percentual de 36,95\% do total
  de municípios com meta estabelecida.
\end{itemize}

\section{Paraná}\label{paranuxe1}

\begin{itemize}
\item
  A capital do Paraná, Curitiba, apresentou uma redução de -1,21\% no
  número de mortes em 2023.
\item
  Dos 399 municípios do Paraná, foram considerados 375 municípios não
  outliers e que tiveram uma meta estabelecida pelo PNATRANS em 2021.
\item
  O estado do Paraná registrou, no total, 2.651 mortes no trânsito em
  2023, apontando um aumento de 8,87\% em relação aos anos 2018-2020,
  que tiveram uma média de 2.435 mortes no trânsito.
\item
  A meta de redução para o Paraná era de -33,33\%.
\item
  114 municípios já atingiram a meta, um percentual de 30,40\% do total
  de municípios com meta estabelecida.
\end{itemize}

\section{Pernambuco}\label{pernambuco}

\begin{itemize}
\item
  A capital de Pernambuco, Recife, apresentou um aumento de 20,19\% no
  número de mortes em 2023.
\item
  Dos 185 municípios do Pernambuco, foram considerados 182 municípios
  não outliers e que tiveram uma meta estabelecida pelo PNATRANS em
  2021.
\item
  O estado de Pernambuco registrou, no total, 1.632 mortes no trânsito
  em 2023, apontando um aumento de 5,47\% em relação aos anos 2018-2020,
  que tiveram uma média de 1.547,33 mortes no trânsito.
\item
  A meta de redução para Pernambuco era de -37,74\%.
\item
  54 municípios já atingiram a meta, um percentual de 29,67\% do total
  de municípios com meta estabelecida.
\end{itemize}

\section{Piauí}\label{piauuxed}

\begin{itemize}
\item
  A capital do Piauí, Teresina, apresentou uma redução de -17,02\% no
  número de mortes em 2023, sendo a segunda capital com melhor
  desempenho.
\item
  Dos 224 municípios do Piauí, foram considerados 204 municípios não
  outliers e que tiveram uma meta estabelecida pelo PNATRANS em 2021.
\item
  O estado do Piauí registrou, no total, 1.045 mortes no trânsito em
  2023, apontando um aumento de 9,58\% em relação aos anos 2018-2020,
  que tiveram uma média de 953,67 mortes no trânsito.
\item
  A meta de redução para o Piauí era de -52,16\%.
\item
  73 municípios já atingiram a meta, um percentual de 35,78\% do total
  de municípios com meta estabelecida.
\end{itemize}

\section{Rio de Janeiro}\label{rio-de-janeiro}

\begin{itemize}
\item
  A capital do Rio de Janeiro, Rio de Janeiro, apresentou um aumento de
  15,09\% no número de mortes em 2023.
\item
  Dos 92 municípios do Rio de Janeiro, foram considerados 90 municípios
  não outliers e que tiveram uma meta estabelecida pelo PNATRANS em
  2021.
\item
  O estado do Rio de Janeiro registrou, no total, 1.965 mortes no
  trânsito em 2023, apontando um aumento de 10,31\% em relação aos anos
  2018-2020, que tiveram uma média de 1.781,33 mortes no trânsito.
\item
  A meta de redução para o estado do Rio de Janeiro era de -19.93\%.
\item
  25 municípios já atingiram a meta, um percentual de 27,78\% do total
  de municípios com meta estabelecida.
\end{itemize}

\section{Rio Grande do Norte}\label{rio-grande-do-norte}

\begin{itemize}
\item
  A capital do Rio Grande do Norte, Natal, apresentou uma redução de
  0,30\% no número de mortes em 2023.
\item
  Dos 167 municípios do Rio Grande do Norte, foram considerados 138
  municípios não outliers e que tiveram uma meta estabelecida pelo
  PNATRANS em 2021.
\item
  O estado de Rio Grande do Norte registrou, no total, 404 mortes no
  trânsito em 2023, apontando uma redução de -14,35\% em relação aos
  anos 2018-2020, que tiveram uma média de 471,67 mortes no trânsito.
\item
  A meta de redução para o Rio Grande do Norte era de -52,17\%.
\item
  67 municípios já atingiram a meta, um percentual de 48,55\% do total
  de municípios com meta estabelecida.
\end{itemize}

\section{Rio Grande do Sul}\label{rio-grande-do-sul}

\begin{itemize}
\item
  A capital do Rio Grande do Sul, Porto Alegre, apresentou um aumento de
  14,2\% no número de mortes em 2023.
\item
  Dos 497 municípios do Rio Grande do Sul, foram considerados 410
  municípios não outliers e que tiveram uma meta estabelecida pelo
  PNATRANS em 2021.
\item
  O estado de Rio Grande do Sul registrou, no total, 1.717 mortes no
  trânsito em 2023, apontando um aumento de 4,87\% em relação aos anos
  2018-2020, que tiveram uma média de 1.637,33 mortes no trânsito.
\item
  A meta de redução para o Rio Grande do Sul era de -27,38\%.
\item
  197 municípios já atingiram a meta, um percentual de 47,80\% do total
  de municípios com meta estabelecida.
\end{itemize}

\section{Rondônia}\label{ronduxf4nia}

\begin{itemize}
\item
  A capital de Rondônia, Porto Velho, apresentou um aumento de 17,05\%
  no número de mortes em 2023.
\item
  Dos 52 municípios de Rondônia, foram considerados 29 municípios não
  outliers e que tiveram uma meta estabelecida pelo PNATRANS em 2021.
\item
  O estado de Rondônia registrou, no total, 459 mortes no trânsito em
  2023, apontando um aumento de 17,49\% em relação aos anos 2018-2020,
  que tiveram uma média de 390,67 mortes no trânsito.
\item
  A meta de redução para Rondônia era de -38,75\%.
\item
  13 municípios já atingiram a meta, um percentual de 26,53\%do total de
  municípios com meta estabelecida.
\end{itemize}

\section{Roraima}\label{roraima}

\begin{itemize}
\item
  A capital de Roraima, Boa Vista, apresentou um aumento de 26,03\% no
  número de mortes em 2023, sendo a capital com o pior desempenho.
\item
  Todos os 15 municípios de Roraima foram considerados não outliers e
  tiveram uma meta estabelecida pelo PNATRANS em 2021.
\item
  O estado de Roraima registrou, no total, 154 mortes no trânsito em
  2023, apontando um aumento de 33,14\% em relação aos anos 2018-2020,
  que tiveram uma média de 115,67 mortes no trânsito.
\item
  A meta de redução para Roraima era de -59,47\%.
\item
  2 municípios já atingiram a meta, um percentual de 13,33\% do total de
  municípios com meta estabelecida.
\end{itemize}

\section{Santa Catarina}\label{santa-catarina}

\begin{itemize}
\item
  A capital de Santa Catarina, Florianópolis, apresentou uma redução de
  -3,57\% no número de mortes em 2023.
\item
  Dos 295 municípios do Amapá, foram considerados 260 municípios não
  outliers e que tiveram uma meta estabelecida pelo PNATRANS em 2021.
\item
  O estado de Santa Catarina registrou, no total, 1420 mortes no
  trânsito em 2023, apontando um aumento de 1,67\% em relação aos anos
  2018-2020, que tiveram uma média de 1396,67 mortes no trânsito.
\item
  A meta de redução para o Amapá era de -34,25\%.
\item
  92 municípios já atingiram a meta, um percentual de 35,38\%\% do total
  de municípios com meta estabelecida.
\end{itemize}

\section{São Paulo}\label{suxe3o-paulo}

\begin{itemize}
\item
  A capital de São Paulo, São Paulo, apresentou uma redução de 50,10\%
  no número de mortes em 2023, sendo a capital que obteve o melhor
  desempenho.
\item
  Dos 645 municípios de São Paulo, foram considerados 582 municípios não
  outliers e que tiveram uma meta estabelecida pelo PNATRANS em 2021.
\item
  O estado de São Paulo registrou, no total, 4.864 mortes no trânsito em
  2023, apontando uma redução de -0,75\% em relação aos anos 2018-2020,
  que tiveram uma média de 4.900,67 mortes no trânsito.
\item
  A meta de redução para o estado de São Paulo era de -26,55\%.
\item
  216 municípios já atingiram a meta, um percentual de 37,11\% do total
  de municípios com meta estabelecida.
\end{itemize}

\section{Sergipe}\label{sergipe}

\begin{itemize}
\item
  A capital de Sergipe, Aracaju, apresentou um aumento de 24,05\% no
  número de mortes em 2023.
\item
  Dos 75 municípios do Amapá, foram considerados 73 municípios não
  outliers e que tiveram uma meta estabelecida pelo PNATRANS em 2021.
\item
  O estado de Sergipe registrou, no total, 446 mortes no trânsito em
  2023, apontando um aumento de 12,53\% em relação aos anos 2018-2020,
  que tiveram uma média de 396,33 mortes no trânsito.
\item
  A meta de redução para Sergipe era de -41,86\%.
\item
  25 municípios já atingiram a meta, um percentual de 34,25\% do total
  de municípios com meta estabelecida.
\end{itemize}

\section{Tocantins}\label{tocantins}

\begin{itemize}
\item
  A capital do Tocantins, Palmas, apresentou um aumento de 12,38\% no
  número de mortes em 2023.
\item
  Dos 139 municípios do Amapá, foram considerados 129 municípios não
  outliers e que tiveram uma meta estabelecida pelo PNATRANS em 2021.
\item
  O estado de Tocantins registrou, no total, 572 mortes no trânsito em
  2023, apontando um aumento de 18,51\% em relação aos anos 2018-2020,
  que tiveram uma média de 482,67 mortes no trânsito.
\item
  A meta de redução para o Tocantins era de -48,91\%.
\item
  50 municípios já atingiram a meta, um percentual de 38,76\% do total
  de municípios com meta estabelecida.
\end{itemize}

\bookmarksetup{startatroot}

\chapter*{Referências}\label{referuxeancias}
\addcontentsline{toc}{chapter}{Referências}

\markboth{Referências}{Referências}

\phantomsection\label{refs}
\begin{CSLReferences}{0}{1}
UFPR; ONSV (2021). O Plano Nacional de Redução Mortes e Lesões no
Trânsito: O Papel dos Municípios. Universidade Federal do Paraná e
Observatório Nacional de Segurança Viária. 53 p.~Disponível em:
\url{https://www.onsv.org.br/pdi/livro-pnatrans-o-papel-dos-municipios}.

ONSV (2024). Base de dados roadtrafficdeaths. Observatório Nacional de
Segurança Viária. Disponível em:
\url{https://github.com/pabsantos/roadtrafficdeaths}.

BRASIL. Ministério dos Transportes. Secretaria Nacional de Trânsito
(Senatran). Frota de veículos 2024. Disponível em:
\url{https://www.gov.br/transportes/pt-br/assuntos/transito/conteudo-Senatran/frota-de-veiculos-2024}.

IBGE (2023). Relação da população dos municípios para publicação no TCU
em 2023. Disponível em:
\url{https://www.ibge.gov.br/estatisticas/sociais/populacao/37734-relacao-da-populacao-dos-municipios-para-publicacao-no-tcu.html}.

\end{CSLReferences}




\end{document}
